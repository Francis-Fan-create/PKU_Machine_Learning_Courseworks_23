\documentclass[12pt, a4paper, oneside]{ctexart}
\usepackage{amsmath, amsthm, amssymb, bm, graphicx, hyperref, mathrsfs}

\title{\textbf{机器学习 第一周作业}}
\author{樊泽羲 2200010816}
\date{\today}
\linespread{1.5}
\newcounter{problemname}
\newenvironment{problem}{\stepcounter{problemname}\par\noindent\textbf{题目\arabic{problemname}. }}{\\\par}
\newenvironment{solution}{\par\noindent\textbf{解答. }}{\\\par}
\newenvironment{note}{\par\noindent\textbf{题目\arabic{problemname}的注记. }}{\\\par}

\begin{document}

\maketitle

\begin{problem}
    简要论述交叉验证法与自助法的异同
\end{problem}

\begin{solution}
    \par
    交叉验证法:分为简单交叉验证法与k折交叉验证法。前者通过给定T'集与V集的数量或比例
    对原始的T集进行随机划分,再进行训练,通过多次训练的R取平均值得到模型的测评结果。后者通过
    在模型训练前把T随机分为k个互不相交、大小相似的子集,每次取其中的一个作为V,其余用作此次训练
    的T'。当V遍历k份数据中的每一个后,将这k次的经验误差取均值作为评估结果
    \par
    自助法:固定训练集大小$\mid T \mid$,对训练集进行有放回抽样$\mid T \mid $次,每次将结果拷贝到
    新的训练集T',并将T中未被抽中的部分分出作为V,把T'上训练的网络在V上的R作为此次训练的误差,多次重复
    的误差取平均值作为测评结果
    \par
    共同点:
    \par\; 1.都是重抽样方法;
    \par\; 2.都可以用于选择合适的模型参数或结构;
    \par\; 3.都需要多次采样取结果的平均值;
    \par\; 4.都通过划分来评估模型的性能
    \par
    不同点:
    \par\; 1.抽样方式不同:交叉验证法是无放回抽样,而自助法是有放回抽样;
    \par\; 2.训练数据的结构不同:交叉验证法的训练集规模小于$\mid T \mid$ ,且不会
       改变数据的分布,而自助法中的训练集可能会包含T中的同一个数据点多次,导致
       数据的分布改变;
    \par\; 3.对$h_T$估计的准确性不同:自助法的T'规模达到了$\mid T \mid$,对$h_T$的
         估计相较交叉验证法更加精准;
    \par\; 4.交叉验证法更适用于数据集较大的情况下,能够更准确地评估模型的性能和泛化能力,自助法则更适用于数据集较小的情况下,能够更高效地利用数据进行模型评估    
\end{solution}


\end{document}
